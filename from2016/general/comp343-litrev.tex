\documentclass[nobib]{tufte-handout}
\usepackage{amsmath}
\usepackage{hyperref}
\usepackage{booktabs}

\title{COMP 343: Literature Survey}
\author{  }
\date{Fall 2016}

\begin{document}
\maketitle

\begin{abstract}
A one semester course in Artificial Intelligence can only scratch the surface of the subject. To provide you with the opportunity to explore AI beyond what is covered in the course, you will carry out a literature review relative to seminal work not addressed in class.
\end{abstract}

\section{Survey Topics, Seminal Work, and Auxiliary Papers}

Your survey topic can cover any topic in AI not discussed in class or plumb the depths of a topic  beyond what's covered in class. Either way, the topic must be approved by the instructor and cannot be duplicated by another member of the class. Your survey must be anchored by one or two seminal works in artificial intelligence.  This means that the survey itself is about the seminal paper, the research that led to the paper, and the research that followed the paper.  It's not just about a topic in AI it's about a topic in AI as it sprung forth in the literature. The emphasis on a seminal paper should ensure that your survey is not a random smattering of work within a topic but a cohesive thread of research within a topic. Your survey should include 8--12 papers in addition to the seminal paper. Exceptions to these bounds will be made on a case by case basis.

\section{The Paper}

Your paper should clearly draw out the connective thread running through the papers surveyed paying special attention to the role played by the seminal work. It must not be a series of individual paper summaries concatenated together into a single ``paper''. Numerous survey papers exist in the literature. The ACM has an entire publication dedicated to them\sidenote{CSUR \url{https://dl.acm.org/citation.cfm?id=J204}}.  You are strongly encouraged to examine existing surveys to determine the structure and organization of a literature survey. Your survey should fall within the 15--20 page range. Exceptions to these bounds will be made on a case by case basis.

\section{The Presentation}

The presentation should provide a brief overview of your survey highlighting the major elements of the topic and the role of the seminal work. It should fall within the 7--10 minute range.

\section{Logistics}

\begin{tabular}{ll}
  \textbf{Papers Surveyed} & Seminal Work + 8--10 Related Papers \\
  \textbf{Page Count} & 15--20 \\
  \textbf{Presentation Length} & 7--10 minutes \\
  \textbf{Topic Due Date} & 9/30 \\
  \textbf{Preliminary Bib. Due} & 10/21 \\
  \textbf{Paper Due} & 11/22 \\
  \textbf{Presentations} & 12/1--2
\end{tabular}

\end{document}
