\documentclass[]{tufte-handout}
\usepackage{amsmath}
\usepackage{hyperref}
\usepackage{booktabs}

\title{Syllabus \\ COMP 343 Artificial Intelligence}
\author{}
\date{Fall 2016}

\begin{document}
\maketitle

\section{Logistics}
\begin{itemize}
\item \textbf{Where: } Center for Science and Business, Room 303
\item \textbf{When: } MTWF,  1--1:50pm
\item \textbf{Instructor:} James \textit{Logan} Mayfield
\begin{itemize}
\item \textit{Office: } Center for Science and Business, Room 344
\item \textit{Phone: } 309-457-2200 % chktex 8
\item \textit{Email: } lmayfield \textit{at} monmouthcollege \textit{dot} edu
\item \textit{Website: } \url{https://jlmayfield.github.io/}
\item \textit{Office Hours: } M 9--10am. Tu 10--11am. F 2--3pm. By Appointment.
\end{itemize}
\item \textbf{Course Website: } \url{https://jlmayfield.github.io/MC-COMP343}
\item \textbf{Credits: } 1 course credit
\end{itemize}
\emph{Note: Parts of this Syllabus are subject to change based on specific class needs.}

\section{Texts}
%insert text #1

\noindent Stuart Russell and Peter Norvig. 2009. \textit{Artificial Intelligence: A Modern Approach (3rd ed.)}. Prentice Hall Press, Upper Saddle River, NJ, USA\@.

\section{Description, Content, and Goals}

In this course we undertake a rigorous study of Artificial Intelligence (AI) in which we
operationalize the task of building an \textit{Artificially Intelligent Agent}. Our
ultimate goal is to know what it means to pursue an AI solution to a problem, how
to pose that problem and its solution in an objective, measurable fashion, and
survey the classic techniques used in AI today.

This course will utilize chapters from parts I, II, IV, and V of the text. Topics covered include:
\begin{itemize}
\item AI History (Chapter 1) %paper
\item The Intelligent Agent Approach (Chapter 2) %paper
\item Problem-Solving as Search (Chapter 3) %program
\item Local Search (Chapter 4) %paper
\item Adversarial Search (Chapter 5) %paper
\item Probability and Uncertainty (Chapter 13) %nothing
\item Probabilistic Reasoning (Chapter 14) %paper
\item Learning from Examples (Chapter 18) %program
\item Learning Probabilistic Models (Chapter 20) %paper
\item Reinforcement Learning (Chapter 21) %program
\end{itemize}
Other sections of the text will be used as needed and as time permits.  Students will have the opportunity to explore topics and chapters beyond this selection through a literature survey project.

A secondary goal of this course is to develop the student's skills for working with primary research papers.  Students can expect to read a dozen or more papers about artificial intelligence by the end of the semester. In addition to reading assigned papers, students will hone their research reading skills through their literature survey on a topic of their choosing.

\section{Programming Language and Environment}

This course has no fixed, required programming language. It is likely that we'll see a mixture of  Python and Matlab/Octave. If a particular language is required or utilized in the course, then it will be made available through one or more of the departmental servers. For programming projects, students will be allowed to use whatever language they choose so long as they are able to carry out a live demo  of their work to the class and instructor.

\section{Expectations and Policies}

You are expected to carry yourself in a mature and professional manner in this course. Towards this end, there are a few classroom policies by which you are expected to abide.
\begin{itemize}

\item \textit{Late Assignments: } In general, late assignments will \textit{not} be accepted.  If you feel you have a justified reason for the assignment being late you may set up an appointment to meet with the instructor and plead your case.  Situations beyond your control are understandable and exceptions can and will be made.

\item \textit{Attendance: } \textbf{Repeated absences and late arrivals to class will quickly reduce your participation grade to zero.}  The occasional late arrival or missed class is one thing, but being habitually late and regularly missing classes is disruptive and not fair to your classmates.

\item \textit{Participation: }  Cellphone and computer usage in class for non-class related activities is strongly discouraged.  All devices should be set to silent when in class.  If your usage of technology becomes a distraction to your classmates or your instructor, then your participation grade will suffer.  If you're not sure if your being a distraction, then err on the side of caution and assume your distracting someone.  Put another way, if the instructor or a classmate has to tell you you're distracting them, then you've already gone too far.

\item \textit{Quality of Work:} There are several minimal requirements that your assignments must meet.
\begin{itemize}
\item \textit{Electronic Submissions:}  When work is submitted electronically, then it is your responsibility to be certain you know and understand the system for doing so and to be sure your work is properly submitted. Not following the instructions for assignment submission can mean your assignment does not get submitted and will be considered late.

\item \textit{Staples:} Assignments that take up more than one page must be stapled.  Unstapled assignments will either be returned to you to be stabled ASAP or points will be deducted.

\item \textit{Neatness:}  Make every attempt to make your work neat and orderly:  label problems, avoid excessive scratching out of mistakes (use pencil if you are prone to errors) and if you use spiral bound paper tear off the edges. Put your name on your work!

\item \textit{Show Work:} Rarely are answers alone sufficient for full credit.  Show your work whenever prudent.  If you're unsure if work is needed, \textit{ask!}
\end{itemize}

\end{itemize}


\subsection{Collaboration}

In general, you are encouraged to make use of the resources available to you.  This means it is OK to seek help from a friend, tutor, instructor, internet, etc.  However, \textit{copying of answers and any act worthy of the label of ``cheating'' is never permissible!}  It is understandable that when you work with a partner or a group that the resultant product is often extremely similar.  This is acceptable but be prepared to be asked to defend your collaborations to the instructor.  \textit{You should always be able to reproduce an answer on your own, and if you cannot you likely \textbf{do not really known the material.}} All of the Monmouth College rules on academic dishonesty apply.  If you violate the rules be prepared to face the consequences of your actions.

\section{Grades}

This courses uses a standard grading scale.  Assignments and final grades will not be curved except in rare cases when its deemed necessary by the instructor.  Percentage grades translate to letter grades as follows:

\begin{center}
\begin{small}
\begin{tabular}{lcl}
Score & & Grade \\ \toprule
94--100 & & A \\
90--93 & & A- \\
88--89 & & B+ \\
82--87 & & B \\
80--81 & & B- \\
78--79 & & C+ \\
72--77 & & C \\
70--71 & & C- \\
68--69 & & D+ \\
62--67 & & D \\
60--61 & & D- \\
0--59 & & F
\end{tabular}
\end{small}
\end{center}

You are always welcome to challenge a grade that you feel is unfair or calculated incorrectly.  Mistakes made in your favor will never be corrected to lower your grade.  Mistakes made not in your favor will be corrected.  \textit{Basically, after the initial grading your score can only go up as the result of a challenge.}

\subsection{Workload}

Graded homework comes in the for of reading assignments in which students must read, summarize and report on research papers from the field of AI\@.  A separate document provides the details of these assignments\sidenote{see course website}.

A problem set of exercises from each chapter covered will be assigned to the class. Some of these problems will be tackled in class and others will show up on exams.  These problems will not be graded but should be completed by the student if for no other reason that they might show up on an exam. There will be five exams throughout the semester. All but the last exam will cover a specific specific set of material. The last exam, given during the finals period, will include at least one comprehensive question. This could be a question from any of the problem sets assigned throughout the semester or a question that stretches across multiple topics in the course.

Students will carry out three programming projects throughout the course of the semester and a deeper literature survey of a subject of their choosing. The literature survey will be presented to the class near the end of the semester.

\begin{center}
  \begin{tabular}{ll}
    Category & Number of Assignments \\
    Reading Assignments & 7 \\
    Programming Projects & 3 \\
    Literature Survey Paper with Presentation & 1 \\
    Exams & 5 \\
  \end{tabular}
\end{center}


\subsection{Grade Weights}

Your final grade is based on a weighted average of particular assignment categories.  You should be able to estimate your current grade based on your scores and these weights.  You may always visit the instructor \textit{outside of class time} to discuss your current standing.
\begin{center}
\begin{tabular}{ll}
  Category & Weight \\ \toprule
  Reading Assignments & 20\% \\ %~3 per
  Programming Projects  & 15\% \\ % 5 per
  Survey Paper + Presentation & 15\% \\ %10 paper, 2 prep, 3 presentation
  Exams & 40 \% \\ % 8% per
  Participation & 10\%
\end{tabular}
\end{center}

\subsection{Course Engagement Expectations}

The weekly workload for this course will vary by student but on average should be about 12.75 hours per week.  The follow tables provides a rough estimate of the distribution of this time over different course components for a 16 week semester.
\begin{center}
\begin{tabular}{lll}
\toprule
Lectures&           & 4 hours/week \\
Reading &           & 2 hours/week \\
Programs & 60 hours  & 4 hours/week \\
Exam Study Time & 24 hours  & 1.5 hours/week \\
Paper/Presentation & 20 hours & 1.25 hours/week \\
\midrule
& & 12.75 hours/week \\
\bottomrule
\end{tabular}
\end{center}


\subsubsection{Calendar}

The following calendar should give you a feel for how work is distributed throughout the semester.  Assignments and events are listed in the week they are due or when the occur. \textit{This calendar is subject to change based on the circumstances of the course.}

\begin{center}
\begin{tabular}{ccr}
\toprule
Week & Dates & Assignments \\
\toprule
1 & 8/23--8/26 & Reading 1 Due. \\
2 & 8/29--9/2 &  Reading 2 Due. \\
3 & 9/5--9/9 &   Reading 3 Due. \\
4 & 9/12--9/16 &  \\
6 & 9/19--9/23 & Program 1 Due. Exam 1.\\
7 & 9/26--9/30  & Lit. Sur. Topic Due. Reading 4 Due. \\
8 & 10/3--10/7 &   \\
9 & 10/10--10/11 &  Reading 5 Due. Exam 2 (Tu). FALL BREAK (W-F). \\
10 & 10/17--10/21 &  Lit. Sur. Bib Due. \\
11 & 10/24--10/28 &  \\
12 & 10/31--11/4 &  Exam 3.\\
13 & 11/7--11/11 &  Reading 6 Due.\\
14 & 11/14--11/18 & Program 2 Due. \\
15 & 11/21--11/22 & Exam 4. THANKSGIVING BREAK (W-F).  Lit. Sur. Paper Due. \\
16 & 11/28--12/2 &  Program 3 Due. Lit. Sur. Presentations\\
17 & 12/5--12/7 & Reading 7 Due.  READING DAY (Th.) \\
\midrule
Final's Week & 12/10 (11:30am-2:30pm) & Exam 5. \\
\bottomrule
\end{tabular}
\end{center}


\end{document}
