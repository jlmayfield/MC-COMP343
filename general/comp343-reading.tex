\documentclass[10pt]{article}
\usepackage{amsmath}
\usepackage{setspace}
\usepackage{hyperref}
\usepackage{booktabs}
\usepackage{bibentry}
\nobibliography*

\setlength{\textheight}{9in} \setlength{\topmargin}{-.5in}
\setlength{\textwidth}{6.5in} \setlength{\oddsidemargin}{0in}
\setlength{\evensidemargin}{0in}

\title{COMP 343 \\ Reading Assignments}
\author{  }
\date{Fall 2016}

\begin{document}
\maketitle

\begin{abstract}
  One goal of this course is to develop your ability navigate the primary sources of computing: journal and conference papers. Towards this end you'll be reading and reporting on at least a paper per chapter covered. This document describes what you need to submit for every paper in order to get credit for the reading as well as how reading assignments will be graded.
\end{abstract}

\section{How to Read}

We'll be using Keshav's three-pass approach in this class\cite{Keshav}. We'll discuss this approach on day one. For the most part, you'll only be doing two-pass readings. A full third-pass will on be necessary for one or two papers relative to your literature review.

\section{What to Turn In}

For each reading assignment you will turn in a one two two page report containing:
\begin{itemize}
  \item \textit{The five Cs}: \newline After the first pass reading you should be able to produce the five Cs. Your focus should first and foremost be on Context and Contributions. Once you understand what's being said in the paper and the context from which it arose, then comment on clarity and correctness. Finally, categorize the paper as best as possible.
  \item \textit{A Summary}: \newline After the second pass you should be able to provide a brief, one to two paragraph summary of the paper's content \textit{in your own words}.
\end{itemize}

\section{Grading}

Reading reports are graded on a three point scale. Scores will be averaged together and
the final reading assignment grade is determined based on the scale below.

\begin{center}
\begin{small}
\begin{tabular}{ll}
Assignment Avg. (Min) & Letter Grade \\ \toprule
2.7   & A  \\
2.5    & A- \\
2.4 & B+ \\
2.1    & B  \\
2   & B- \\
1.7    & C+ \\
1.5 & C  \\
1   & C- \\
0.7    & D  \\
< 0.7  & F
\end{tabular}
\end{small}
\end{center}

\section{Papers}

You'll be reading the following papers in the following order. For due dates see the syllabus. 

\begin{enumerate}
  \item \bibentry{Turing}
  \item \bibentry{Horvitz}
  \item \bibentry{KorfSchultze}
  \item \bibentry{MintonEtAl}
  \item \bibentry{FrankEtAl}
  \item \bibentry{Pearl}
  \item \bibentry{SilverEtAl}
\end{enumerate}

\bibliography{comp343.bib}
\bibliographystyle{acm}
\end{document}
